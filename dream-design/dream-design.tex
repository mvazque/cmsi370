\documentclass[11pt]{article}
\usepackage{fullpage}
\usepackage{graphicx}
\title{Dream User Interface for Web Browsing}
\author{Miguel Vazquez}
\begin{document}
\maketitle
\tableofcontents

\section{Introduction}
Web browsers are a tool that many of us are used to nowadays. However that is not to say that this tool has reached its pinnacle of perfection and it does not mean that it is not what the people need. It should also be mentioned that this is still a tool that can only be used to its maximum potential by a select group of people. This in itself is good except that the select few who use it well have had teachings with it and does not allow for other to do so effectively. 

\section{Components}
There are various ideas that I have related to a user interface for web browsing. They are as follows.

\subsection{Working with Bricks}
Whether a person is young or old they never lose the fascination of building something. The ability to create and then use something can cause for a grand feeling of accomplishment that can cause for a desire to create even more. This feeling of creation is something that makes Legos so popular and is something that is felt whenever one uses them. My idea for this is to allow for simple piecing together and taking apart of various aspects related to the browser. The browser would have different media squares or “bricks.” These would be attachable to anything within the browser and can come in various sizes and uses. The idea is that each of these bricks can act as a browser. By putting multiple together one can view multiple pages at once. Of course this may not cause the most comfort for the user and each media square can be made larger so as to take an entire monitor. This is partially done to improve on the idea of tabs. They are very helpful tools but at times one may have multiple tabs from the same website causing them all to have the same name. This causes the possibility that a user will need to go through some trial and error in order to find the tab they want. The ability to have a window to each individual tab will cause this problem to go away. The next aspect to this brick idea is that a user can combine the bricks to make the screen bigger. This allows for a user to make a video play on a screen they specifically want. The inclusion of color to each brick also allows for some added color to what can sometimes be considered as drab and boring user experience. The hope is that with this customizability the user will be excited to be using their browser even if it isn’t for the sites they will access or the media they will find.

\subsection{Achievement System}
Something that has been appearing more and more with interactive media, namely video games is a system of achievements. Although these systems don’t actually mean anything outside of the media they are found in they do have certain desirability. The achievements or badges can be related to a person’s specific account. The thing about these achievements is that they will be directly related to the usage of the web browser. They will be challenges such as linking bricks in a certain matter or using certain properties of the bricks. However some more challenging achievements will also be made available. The trick to these achievements is that they will have tips and tricks related to them as well as forums for the users to help one another. These trickier achievements will include adding new functionality to the bricks, inspecting the elements of a webpage, and even creating simple HTML webpage. By “challenging,” it is obvious that it is anything related to programming and background elements at play that provide the user with the browsing experience they enjoy. It also comes down to making the user achievement even better. One of the achievements will aimed towards showing the user how to submit any of their ideas or code that they believe should be shared. Of course their own creations can be used by themselves freely but by submitting their work the people in charge of the browser can choose what should be added. The work can also be submitted to the forums so that the users themselves can vote over which user created features they would like to use freely. This will allow for an interesting blend of company induced changes as well as user customizability. It is basically all meant with the users in mind.

\subsection{Improved User Experiencen}
Besides the brick idea that has already been mentioned there a couple more ideas that can easily make the web browsing experience even more enjoyable. One such idea is meant for the loading of the different web pages. This wait time can feel tiresome for people and can often be one of the most difficult and annoying points of the web browsing experience. The idea for this to be fixed is to have the browser know internet speed and the size of the information that need to be received for the site. With this in mind an exact time for loading could be made. As was stated the waiting can be the worst part and not knowing how long one has to wait can make it even worse. The idea is that by telling the user how much time it will take they can do other things to pass their time instead of sitting there. An idea that goes along with this timer is the inclusion a time passing application. This could be a variety of things such as a game, a memo pad, or even something else that the user has created. This will help the user as it will make the wait time for the load feel less tedious and also keep the user on the web browser so that they do not do something else on their computer that could stray them away from the reason they opened their web browser in the first place. This can help cause a more enjoyable experience for the user. The final part of this enhanced experience leans towards the safety of the user. This involves changing the mouse cursor to an arrow with a question mark or some other symbol like that which will clearly tell the user that they should be cautious with the object they plan on selecting. This would also involve asking the user a second time whether they want to have a toolbar attached to the web browser. It is no secret that nowadays most downloads mention that they have a toolbar from there company that they now offer to the user for their browsing experience. Many people download what they need and download the toolbars along with it whether it is because they think it will be a good idea or do not pay attention to what is being downloaded. The second check for the toolbar will show the user what is being offered once more but also show how the performance of the browser will be affected. This means that the speed of the browser, as well the capacities of it is mentioned. These statistics will be listed in a simplified manner that the user can actually understand it. This will allow the user to become familiar with the terminology related to computers and interest them in efficiency of the browsing experience and their own programs.

\section{Scenarios}

%\begin{figure}[h!]
 % \centering
    %\includegraphics[width= 1\textwidth]{./Images/Data_table}
  %\caption{Data collected}
 %\label{Collect}
%\end{figure}

%Figure~\ref{Collect}

\end{document}
